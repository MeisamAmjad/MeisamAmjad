%%%%%%%%%%%%%%%%%%%%%%%%%%%%%%%%%%%%%%%%%%%%%%%
%%%%%%%%%%%%%%%%%%%%%%%%%%%%%%%%%%%%%%%%%%%%%%%
%
%                                                    Packages
%
%Packages extend the basic \LaTeX\ commands and formatting for special situations. amsthm is %for altering the Theorem environments. amsmath implements almost all of the mathematical %symbols. These are included in the woosterthesis class. amssymb adds the mathematical %symbols not present in amsmath. verbatim and alltt print the text just as entered in a fixed %width font and are useful for code examples. Other packages that might be useful are graphicx %(loaded with the woosterthesis class), float (for figures), pdfsync (allows you to click in the %output and be taken to that point in the input), and pxfonts (for the Palatino font which is the %Wooster standard). The woosterthesis class loads the graphicx package with either the pdftex %option or the dvips option accordingly. You can remove the % character to activate the %packages below. Most people probably won't need these.
%
%%%%%%%%%%%%%%%%%%%%%%%%%%%%%%%%%%%%%%%%%%%%%%%
%%%%%%%%%%%%%%%%%%%%%%%%%%%%%%%%%%%%%%%%%%%%%%%

\usepackage{makeidx} % uncomment if you are creating an index
\usepackage{verbatim} % uncomment to use the verbatim environment
\usepackage{lettrine} % uncomment to be able to do dropped caps
%\usepackage[Euler]{styles/woofncychap} % fancy chapter headings
%\usepackage[Gauss]{styles/woofncychap}
%\usepackage{alltt} % uncomment to allow the use of the alltt environment
\usepackage{styles/woolshort} % comment this out for your thesis
%\usepackage{osxwarnockpro}







