\chapter{Classical Oscillator Code}
We show the code in QuTiP to generate Figs.~\ref{fig3Harmonic} and \ref{fig4Thermal}.

\begin{verbatim}
import math
import pylab as pl
from qutip import *

# system parameters
n = 2 # photon excitations
gamma = 1
kappa = 1
Y = 0.01
g = 3
g_M = 2
beta = 1
beta_star = 1

# initial state
psi0 = tensor(fock(2,1), fock(n,0))

# operators of interest
sm = tensor(sigmam(), qeye(n))
a = tensor(qeye(2), destroy(n))

# collapse operators
collapse = [sqrt(2*kappa)*a, sqrt(gamma)*sm]

# expectation values
evalue_list = [a.dag()*a, sm.dag()*sm]

# timescale
harmonic_time = linspace(0,15,7500)
thermal_time = linspace(0,10,6000)

# build the time-dependent Hamiltonian
def H_t(t, args):
    H0 = args[0]
    H1 = args[1]
    omega_M = args[2]

    return H0 + H1 * sin(omega_M*t)

H0 = Y*(a + a.dag()) + g*(a*sm.dag() + a.dag()*sm)
H1 = g_M*a.dag()*a
omega_M = 1

H_args = (H0, H1, omega_M)

# evaluate the expectation values
evalues = odesolve(H_t, psi0, harmonic_time, collapse, evalue_list, H_args)

# write to data file
import csv
filename1 = 'harmonic_oscillator.csv'
with open(filename1, 'wb') as f:
    writer = csv.writer(f)
    writer.writerows(array([harmonic_time, evalues[0], evalues[1]]).transpose())
    
# generate a pseudorandom value for the thermal state
xi = random.random()

# define a new Hamiltonian
H_therm = Y*(a + a.dag()) + g*(a*sm.dag() + a.dag()*sm) + xi*g_M*a.dag()*a

# evaluate the expectation values a second time
evalues_therm = odesolve(H_therm, psi0, thermal_time, collapse, evalue_list)

# write to data file
filename2 = 'thermal_oscillator.csv'
with open(filename2, 'wb') as f:
    writer = csv.writer(f)
    writer.writerows(array([thermal_time, evalues_therm[0], 
    	evalues_therm[1]]).transpose())

# plot the graphs to verify
pl.plot(harmonic_time, evalues[0], 'r', harmonic_time, evalues[1], 'b')
pl.xlabel('gamma*t')
pl.ylabel('expectation value')
pl.legend(('cavity', 'atom'))
pl.show()

pl.plot(thermal_time, evalues_therm[0], 'r', thermal_time, evalues_therm[1], 'b')
pl.xlabel('gamma*t')
pl.ylabel('expectation value')
pl.legend(('cavity', 'atom'))
pl.show()
\end{verbatim}