\chapter{Probe Spectrum Code}
We show the code in QuTiP to generate Fig.~\ref{fig5d}.

\begin{verbatim}
import math
import pylab as pl
from qutip import *

# system parameters
n = 2
m = 64
gamma = 1
kappa = 1
Y = 0.01
g = 3
g_M = 5
omega_M = 1
kappa_M = 0
ntraj = 1

# initial state
psi0 = tensor(fock(2,1), fock(n,0), fock(m,0))

# operators
sig_plus = tensor(sigmap(), qeye(n), qeye(m))
a = tensor(qeye(2), destroy(n), qeye(m))
b = tensor(qeye(2), qeye(n), destroy(m))

# collapse operators
collapse = [sqrt(2*kappa)*a, sqrt(gamma)*sig_plus.dag()]

# expectation values
evalue_list = [a.dag()*a, sig_plus*sig_plus.dag(), b.dag()*b]

# lists of steady-state values
adaga_list = []
spsm_list = []
bdagb_list = []

Delta_list = linspace(-10,10,400)
times_list = linspace(0,10,1000)

# build the probe spectrum
for Delta in Delta_list:
    H = Delta*a.dag()*a + Delta*sig_plus*sig_plus.dag() + omega_M*b.dag()*b 
    	+ Y*(a + a.dag()) + g*((a*sig_plus) + (a.dag()*sig_plus.dag())) 
	+ g_M*a.dag()*a*(b + b.dag())
    evalues = mcsolve(H, psi0, times_list, ntraj, collapse, evalue_list)
    adaga_list.append(evalues[0][-1])
    spsm_list.append(evalues[1][-1])
    bdagb_list.append(evalues[2][-1])

# plot the probe spectrum
pl.plot(Delta_list, adaga_list, 'r', Delta_list, spsm_list, 'b', 
	Delta_list, bdagb_list, 'g')
pl.xlabel('detuning')
pl.ylabel('steady-state values')
pl.legend(('cavity', 'atom', 'oscillator'))
pl.show()
\end{verbatim}