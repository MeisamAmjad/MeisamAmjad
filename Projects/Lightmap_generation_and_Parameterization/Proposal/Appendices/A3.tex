\chapter{Correlation Function Code I}
We show the code in QuTiP to generate Fig.~\ref{fig7g20}.

\begin{verbatim}
import math
import pylab as pl
from qutip import *

# system parameters
n = 3
m = 128
gamma = 1
kappa = 0.5
Y = 0.01
g = 1/sqrt(2.0)
omega_M = 1
kappa_M = 0.1
ntraj = 1

# initial state
psi0 = tensor(fock(2,1), fock(n,0), fock(m,0))

# operators
sig_plus = tensor(sigmap(), qeye(n), qeye(m))
a = tensor(qeye(2), destroy(n), qeye(m))
b = tensor(qeye(2), qeye(n), destroy(m))

# collapse operators
collapse = [sqrt(2*kappa)*a, sqrt(gamma)*sig_plus.dag(), sqrt(kappa_M)*b]

# expectation values
evalue_list = [a.dag()*a.dag()*a*a, a.dag()*a*sig_plus*sig_plus.dag(), a.dag()*a,
 sig_plus*sig_plus.dag()]

# lists of steady-state values
fieldfield_list = []
atomfield_list = []

times_list = linspace(0,20,2000)
gM_list = linspace(0,3,200)

# build the correlation plots
for g_M in gM_list:
    H = omega_M*b.dag()*b + Y*(a + a.dag()) 
    + g*((a*sig_plus) + (a.dag()*sig_plus.dag())) + g_M*a.dag()*a*(b + b.dag())
    evalues = mcsolve(H, psi0, times_list, ntraj, collapse, evalue_list)
    fieldfield_list.append(evalues[0][-1]/(evalues[2][-1]**2))
    atomfield_list.append(evalues[1][-1]/(evalues[2][-1]*evalues[3][-1]))

# plot the correlation functions
pl.plot(gM_list, fieldfield_list, 'r', gM_list, atomfield_list, 'b')
pl.xlabel('g_M')
pl.ylabel('intensity correlation values')
pl.legend(('field-field', 'atom-field'))
pl.show()
\end{verbatim}