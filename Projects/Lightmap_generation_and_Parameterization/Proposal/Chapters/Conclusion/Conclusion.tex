\chapter{Conclusion}
In this thesis, we have investigated the probe spectra and nonclassical photon statistics of a damped, weakly-driven cavity optomechanical system. In order to best understand such results, three principle theoretical techniques are introduced, and a more straightforward classical oscillator in two different initial states is examined. A full quantum treatment calculates the probe spectra under various values for both the optical and mechanical coupling rates, as well as determining second-order correlation functions for the field-field auto-correlation, atom-field, and oscillator-field cross-correlations.

\section{Summary}
In Chapter 2, we illustrate three crucial theoretical frameworks necessary to conduct an investigation of our cOM system. The master equation in Lindblad form is derived in order to describe the dynamics of an open quantum system, and we utilize a system-reservoir approach in the interaction picture, as well as the Born-Markov approximation and some time-dependent perturbation theory, to arrive at the final result. In addition, consideration of the system at zero temperature is shown to further simplify the master equation and instructs our approach to the inclusion of damping terms.

The theory of quantum trajectories is introduced, a way to model a cQED experiment as a scattering interaction and reduce the number of system variables to be monitored from an effectively infinite number to manageable finite quantity. The idea of a measurement record is described, where a record of a given time interval and the various quantum jumps are monitored, and the state of the system is conditioned upon the record obtained. The density operator is written in the basis of measurement records, both with and without known initial states. Trajectories and the master equation are integrated using superoperator formalism, where the latter is comprised of two pieces respectively describing between-jump evolution and the jumps themselves. As a final note on trajectories, the dynamics of pure states are considered in the presence of dissipation, enabling us to write the appropriate jump operator, to understand that an initial pure also ends as a pure state, and to determine the probability for records with and without a jump detected.

Lastly, chapter 2 describes the formalism of, and some contrasts with cQED to, cavity optomechanics. The position operator of the mechanical oscillator, a spring-mounted end mirror of the optical cavity, is written using phonon creation and annihilation operators, and the standard linear optomechanical coupling Hamiltonian is written. The optomechanical coupling rate is itself related to the oscillator in terms of its mass, resonant frequency, and frequency modulation of the cavity mode based on distance from equilibrium. Expectations for the output of the probe spectra are briefly noted from previous work, and the full optomechanical Hamiltonian is written.

Chapter 3 considers a classical mechanical oscillator in the scenarios of (i) sinusoidal and (ii) thermal or Brownian fluctuations. The position operator of the oscillator is cast in classical terms, and expectation values are calculated to their steady state conditions, requiring at least ten atomic spontaneous emission lifetimes. We also justify our choices of parameter values, pointing out experimental precedent.

Chapter 4 brings a fully quantum mechanical treatment of the cOM system, first in the form of the calculation and plotting of various probe spectra. Three principle expectation values are determined in the steady state over a range of incident field detunings, varying first the optomechanical coupling and then the optical Jaynes-Cummings coupling rates. Variation from the expected vacuum Rabi splitting is explored, especially due to the relationship of the two coupling rates. The role of photon statistics in cQED is briefly introduced, followed by calculation and plotting of $g^{(2)}(0)$ for both auto- and cross-correlations within the atom-field subsystem. Finally, the time-dependent correlation $g^{(2)}(\tau)$ is calculated, with an explanation of the process involved in doing so, which shows evidence of anharmonicity.

\section{Future Work}
A few avenues of future investigation into this cavity optomechanical system become apparent in this thesis, first of which is the optomechanical Hamiltonian itself. While we consider the standard linear coupling, other configurations are possible such as quadratic coupling \cite{li2012, agarwal2011}. Such a change would require a recalculation of all probe spectra and correlation functions in order to determine the effect. The coupling parameters may be varied over larger intervals in finer steps in order to study their signature on the system, especially through the probe spectra. The effect of temperature on the movable end mirror, as well as the oscillator linewidth $\kappa_M$, is also left mostly unexplored. Similarly, it seems prudent to consider larger number of phonon excitations in calculating the various correlation functions. When examining the latter, additional functions could be calculated such as additional auto- and cross-correlations involving the atom and oscillator. Lastly, deeper study into the physical processes of the system will be required in order to explain the features of the many plots in Chapter 4, namely the peaks of the probe spectra and various oscillatory behaviors of the correlation functions.